\def\paperversiondraft{draft}
\def\paperversionblind{blind}

\ifx\paperversion\paperversionblind
\else
  \def\paperversion{blind}
\fi

% !TEX TS-program = pdflatex
% !TEX encoding = UTF-8 Unicode

% This is a simple template for a LaTeX document using the "article" class.
% See "book", "report", "letter" for other types of document.

\documentclass[12pt]{article} % use larger type; default would be 10pt

\usepackage{longtable}
\usepackage{booktabs}
\usepackage{xargs}
\usepackage{xparse}
\usepackage{xifthen, xstring}
\usepackage{ulem}
\usepackage{xspace}
\usepackage{multirow}
\setlength {\marginparwidth }{2cm}
\usepackage{todonotes}
%% \bibliographystyle{amsalpha}
\makeatletter
\font\uwavefont=lasyb10 scaled 652
\DeclareSymbolFontAlphabet{\mathrm}    {operators}
\DeclareSymbolFontAlphabet{\mathnormal}{letters}
\DeclareSymbolFontAlphabet{\mathcal}   {symbols}
\DeclareMathAlphabet      {\mathbf}{OT1}{cmr}{bx}{n}
\DeclareMathAlphabet      {\mathsf}{OT1}{cmss}{m}{n}
\DeclareMathAlphabet      {\mathit}{OT1}{cmr}{m}{it}
\DeclareMathAlphabet      {\mathtt}{OT1}{cmtt}{m}{n}
%% \newcommand\colorwave[1][blue]{\bgroup\markoverwith{\lower3\p@\hbox{\uwavefont\textcolor{#1}{\char58}}}\ULon}
% \makeatother

% \ifx\paperversion\paperversiondraft
% \newcommand\createtodoauthor[2]{%
% \def\tmpdefault{emptystring}
% \expandafter\newcommand\csname #1\endcsname[2][\tmpdefault]{\def\tmp{##1}\ifthenelse{\equal{\tmp}{\tmpdefault}}
%    {\todo[linecolor=#2!20,backgroundcolor=#2!25,bordercolor=#2,size=\tiny]{\textbf{#1:} ##2}}
%    {\ifthenelse{\equal{##2}{}}{\colorwave[#2]{##1}\xspace}{\todo[linecolor=#2!10,backgroundcolor=#2!25,bordercolor=#2]{\tiny \textbf{#1:} ##2}\colorwave[#2]{##1}}}}}
% \else
% \newcommand\createtodoauthor[2]{%
% \expandafter\newcommand\csname #1\endcsname[2][\@nil]{}}
% \fi


%%% Examples of Article customizations
% These packages are optional, depending whether you want the features they provide.
% See the LaTeX Companion or other references for full information.

%%% PAGE DIMENSIONS
\usepackage{geometry} % to change the page dimensions
\geometry{a4paper} % or letterpaper (US) or a5paper or....
\geometry{margin=1in} % for example, change the margins to 2 inches all round
% \geometry{landscape} % set up the page for landscape
%   read geometry.pdf for detailed page layout information

\usepackage{graphicx} % support the \includegraphics command and options

% \usepackage[parfill]{parskip} % Activate to begin paragraphs with an empty line rather than an indent

\usepackage[utf8x]{inputenc}
\usepackage{amssymb}
\usepackage{listings}

\usepackage{color}
\definecolor{keywordcolor}{rgb}{0.7, 0.1, 0.1}   % red
\definecolor{commentcolor}{rgb}{0.4, 0.4, 0.4}   % grey
\definecolor{symbolcolor}{rgb}{0.0, 0.1, 0.6}    % blue
\definecolor{sortcolor}{rgb}{0.1, 0.5, 0.1}      % green
\usepackage{listings}


%%% PACKAGES
\usepackage{inputenc}
\usepackage{booktabs} % for much better looking tables
\usepackage{array} % for better arrays (eg matrices) in maths
\usepackage{paralist} % very flexible & customisable lists (eg. enumerate/itemize, etc.)
\usepackage{verbatim} % adds environment for commenting out blocks of text & for better verbatim
\usepackage{subfig} % make it possible to include more than one captioned figure/table in a single float

\usepackage{textcomp}


% These packages are all incorporated in the memoir class to one degree or another...

%%% HEADERS & FOOTERS
\usepackage{fancyhdr} % This should be set AFTER setting up the page geometry
\pagestyle{fancy}
\renewcommand{\headrulewidth}{0pt} % customise the layout...
\lhead{\leftmark}\chead{}\rhead{\rightmark}
\lfoot{}\cfoot{\thepage}\rfoot{}

%%% SECTION TITLE APPEARANCE
\usepackage{sectsty}
\allsectionsfont{\sffamily\mdseries\upshape} % (See the fntguide.pdf for font help)
% (This matches ConTeXt defaults)

%%% ToC (table of contents) APPEARANCE
\usepackage[nottoc,notlof,notlot]{tocbibind} % Put the bibliography in the ToC
\usepackage[titles,subfigure]{tocloft} % Alter the style of the Table of Contents
\renewcommand{\cftsecfont}{\rmfamily\mdseries\upshape}
\renewcommand{\cftsecpagefont}{\rmfamily\mdseries\upshape} % No bold!

\setlength{\parindent}{0em}
\setlength{\parskip}{1em}
\usepackage{amsmath}
\usepackage{amsthm}
\usepackage{upgreek}
\usepackage{tikz-cd}
\theoremstyle{definition}
\newtheorem{thm}{Theorem}[subsection]
\theoremstyle{definition}
\newtheorem{corol}[thm]{Corollary}
\theoremstyle{definition}
\newtheorem{lemma}[thm]{Lemma}
\theoremstyle{definition}
\newtheorem{defn}[thm]{Definition}
\newtheorem{exmpl}[thm]{Example}
\usepackage{lscape}
\usepackage{hyperref}
\usepackage{titlesec}

\setcounter{secnumdepth}{4}

\titleformat{\paragraph}
{\normalfont\normalsize}{\theparagraph}{1em}{}
\titlespacing*{\paragraph}
{0pt}{3.25ex plus 1ex minus .2ex}{1.5ex plus .2ex}

%%% END Article customizations

%%% The "real" document content comes below...

\title{Notes on using Representable Functors in Formalised Mathematics}
\author{Christopher Hughes}

\begin{document}


\section{Definition of Representable Functor}

An object $X$ in a category $C$ is a representation of a functor $F : C^{op} \rightarrow Set$ if there is a natural isomorphism

\begin{equation}
  \text{Hom}(-, X) \cong F
\end{equation}

Precisely what this means is that for any object $Y$ in $C$ there is an isomorphism $\phi_Y$ between the set $\text{Hom}(X, Y)$ and the set $F(Y)$ and
that this isomorphism satisfies the following equations.

For any $Y$ and $Z$ in $C$, and any $f \in \text{Hom}(Z, Y)$ and $g$ in $\text{Hom}(Y, X)$, we have
\begin{equation}
  \phi_Z(f ; g) = F(f)(\phi_Y(g))
\end{equation}

This can also be expressed as the commutativity of the following square

% https://tikzcd.yichuanshen.de/#N4Igdg9gJgpgziAXAbVABwnAlgFyxMJZABgBpiBdUkANwEMAbAVxiRAB12cYAPHYABIQAtgF8AFAE1SAAgAaAShCjS6TLnyEUZAIxVajFm07c+gkRIBasxctUgM2PASI7y++s1aIQAMSlKKmpOmq6ketSeRj7+loH6MFAA5vBEoABmAE4iSGQgOBBIbiAMdABGMAwACurOWiCZWEkAFjggkYbeIOkA3HYZ2cK51AVIAEwdXsbsaM1YAPqS-d2DRSOFiADM1KUV1bWhPo0tbZPRHDNz85bLWTmIE-kb2yXllTUhLj4MMOmnBlMYuJ0vFREA
\begin{tikzcd}
  {\text{Hom}(Y, X)} \arrow[d, "f;"'] \arrow[r, "\phi_Y"] & F(Y) \arrow[d, "F(f)"] \\
  {\text{Hom}(Z, X)} \arrow[r, "\phi_Z"']                 & F(Z)
\end{tikzcd}


This above equation is equivalent to assuming naturality of $\phi^{-1}$ which is
For any $Y$ and $Z$ in $C$, and any $f \in \text{Hom}(Z, Y)$ and $y \in F(Y)$, we have
\begin{equation}
  \phi_Z^{-1}(F(f)(y)) = f ; \phi_Y^{-1}(y)
\end{equation}

This can be expressed as the commutativity of the following square

% https://tikzcd.yichuanshen.de/#N4Igdg9gJgpgziAXAbVABwnAlgFyxMJZARgBoAGAXVJADcBDAGwFcYkQAdDnGADx2AAJCAFsAvgAoAmqQAEADQCUIMaXSZc+QijLFqdJq3Zce-IaMkAtOUpVqQGbHgJFyFfQxZtEIAGLTlVXUnLVdSPRpPIx9-WzF9GCgAc3giUAAzACdRJDcQHAgkMhBGegAjGEYABQ1nbRKYdJwQSMNvEHSAbjsM7JEkACYaAtzWr2MONAALLAA9YABaYjEAfSkejr7B4cLEAGYx6L8JdOUaUorq2tCfTKwkqeagzZz9naLz8sqakJdb+8eLQM4x8XGmc0WyxWlhUlDEQA
\begin{tikzcd}
  F(Y) \arrow[r, "\phi^{-1}_Y"] \arrow[d, "F(f)"'] & {\text{Hom}(Y, X)} \arrow[d, "f;"] \\
  F(X) \arrow[r, "\phi^{-1}_Z"']                   & {\text{Hom}(Z, X)}
\end{tikzcd}

One useful characterisation of a representation of a functor is the following. $X$ is a representation of
whenever there is the following

\begin{itemize}
  \item An element $c$ of $F(X)$
  \item For every object $Y$ of $\mathcal{X}$, a map $r_Y : F(X) \to \mathcal{X}(Y, X)$
  \item For any object $Y$ of $\mathcal{X}$, and every element $y \in F(Y)$, $F(r_Y(y))(c) = f$
  \item For any two morphisms $f, g : \mathcal{X}(Y, X)$, if $F(f)(c) = F(g)(c)$ then $f = g$
\end{itemize}

The above functions are enough to define a natural isomorphism between $\text{Hom}(-, X)$ and $F$.
With $f \in \text{Hom}(Y,X)$, we can define $\phi_Y(f)$ to be $F(f)(c)$ and
 for any $y \in F(Y)$, we can define $\phi_Y^{-1}(y)$ to be $r_Y(y)$.

A corepresentation of a functor $F : C \rightarrow Set$ is defined in a similar way.
An object $X$ is a corepresentation of $F$ if there is a natural isomorphism

\begin{equation}
  \text{Hom}(X, -) \cong F
\end{equation}

This can be characterised by the following data about $F$

\begin{itemize}
  \item An element $u$ of $F(X)$
  \item For every object $Y$ of $\mathcal{C}$, a map $e_Y : F(Y) \to \mathcal{C}(X, Y)$
  \item For any object $Y$ of $\mathcal{C}$, and every element $f : F(Y)$, $F(e_Y(f))(u) = f$
  \item For any two morphisms $f, g : \mathcal{C}(X, Y)$, if $F(f)(u) = F(g)(u)$ then $f = g$
\end{itemize}


There are many examples of representations or corepresentations of functors in mathematics.

If $C$ is a category, and $X$ and $Y$ are objects of $C$, then the product, $X \times Y$ is a representation
of the functor $Hom(-, X) \times Hom(-, Y)$, meaning for any object $Z$ of $C$, the set
$\text{Hom}(Z, X \times Y)$ is isomorphic to the set $\text{Hom}(Z, X) \times \text{Hom}(Z, Y)$.

In the category of sets, the set with one element, $1$, is the terminal object. This means it is a representation
of the constant functor $1 : Set \to Set$, i.e. the set of morphisms from any set $X$ into $1$ has exactly
one element.

In the category of sets, the empty set, $\emptyset$, is the initial object. This means it is a corepresentation
of the constant functor $1 : Set \to Set$, i.e. the set of morphisms from $\emptyset$ into any set $X$ has
exactly one element.

In the category of commutative rings, the polynomial ring $R[X]$ is a representation of the functor
$\text{Hom}(R, -) \times \text{Forget}$, where $\text{Forget}$ is the forgetful
functor from commutative rings to sets. This means that for any commutative ring $S$ a ring homomorphism
$R[X] \to S$ can be constructed by taking a ring homomorphism $R \to S$ and an element of $S$.

In the category


\section{Examples of Equality Proofs Using UMP}

Using universal properties provides a convenient way of defining and proving equalities of morphisms in
categories.

\section{Examples of Isomorphisms made easier with UMP}


\section{Inheritance of UMP}

\subsection{Grothendieck Construction, Categories of Elements and Pi Categories}

\subsection{Monoidal Closed Categories}

\subsection{Maybe Something about Adjoints Preserving (co)limits}

\subsection{Something About Parametricity}

\section{Comparison with UMP of Inductive Types}

\section{The Isomorphism Problem - Transferring across Isomorphisms}

\end{document}